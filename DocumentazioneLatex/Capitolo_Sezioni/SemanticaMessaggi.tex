I messaggi inviati dai client sono stringhe che possono essere identificate dalla seguente \textit{RegEx} (\textit{Regular Expression}): 
\begin{center}
	\begin{lstlisting}
	^[a-zA-Z0-9]+;[a-zA-Z0-9]+:[0-9]+;([a-zA-Z0-9]+:[a-zA-Z0-9]+;)+$
	\end{lstlisting}
\end{center}
la cui semantica può essere così rappresentata:
\begin{center}
"<username>;requestType:<tipo>;<richiesta>"
\end{center}
I valori assunti da <tipo> sono i seguenti e corrispondo ai campi definiti all'interno dell'enumerativo \textit{RequestType}.
Durante l'attuale iterazione siamo andati a gestire le richieste relative alle sole due seguenti tipologie di richieste:
\begin{itemize}
	\item 0 -> BOOK\_REGISTRATION\_MANUAL
	\item 8 -> BOOK\_SEARCH
\end{itemize}

Nella seconda iterazione, e nelle successive, è stata pianificata l'implementazione della parte di gestione delle restanti richieste:

\begin{itemize}
	\item 1 -> BOOK\_RESERVATION	
	\item 2 -> LOGIN
	\item 3 -> SIGN\_IN
	\item 4 -> BOOK\_REGISTRATION\_AUTOMATIC
	\item 5 -> PROFILE\_INFO
	\item 6 -> TAKEN\_BOOKS
	\item 7 -> PICK\_UP	
\end{itemize}
La richiesta inviata dal client Android verso il server, ad esempio, per la ricerca di un libro assume la seguente forma:
\begin{center}
	\begin{lstlisting}
	username + ";" + "requestType:" + 0 + ";" + book.encode();
	\end{lstlisting}
\end{center}
\noindent Lato Android, la richiesta vera e propria è preceduta dall'username dell'utente collegato in modo che lato server, qualora, per diversi motivi, l'username fosse invalido, la richiesta venga ignorata a priori.
\\ \noindent
Analogamente possiamo identificare i messaggi inviati dal server come:
\begin{center}
	\begin{lstlisting}
	^[a-zA-Z0-9]+:[0-9]+;([a-zA-Z0-9]+:[a-zA-Z0-9]+;)+$
	\end{lstlisting}
\end{center}
la cui semantica può essere così rappresentata:
\begin{center}
	"requestType:<tipo>;<risultato>"
\end{center}

\noindent Ad esempio, nel caso di invio di una richiesta da parte del server verso lato Android in merito alla registrazione manuale di un libro andata a buon fine, la stringa assume il seguente formato:
\begin{center}
	\begin{lstlisting}
		"requestType:0;result:" + 1 + ";BCID:" + bcid
	\end{lstlisting}
\end{center}
dove \textit{bcid} è il codice alfanumerico generato casualmente dal server per identificare il libro all'interno della rete di Book Crossing.
Qual'ora, invece, il server riceva una richiesta con un tipo errato la risposta del server è la seguente:
\begin{center}
	\begin{lstlisting}
	"requestType:10000;result:KO_RequestType"
	\end{lstlisting}
\end{center}




