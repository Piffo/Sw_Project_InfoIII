I messaggi inviati dai client sono stringhe che possono essere identificate dalla seguente \textit{RegEx} (\textit{Regular Expression}): 
\begin{center}
	\begin{lstlisting}
	^[a-zA-Z0-9]+;[a-zA-Z0-9]+:[0-9]+;([a-zA-Z0-9]+:[a-zA-Z0-9]+;)+$
	\end{lstlisting}
\end{center}
la cui semantica può essere così rappresentata:
\begin{center}
"<username>;requestType:<tipo>;<richiesta>"
\end{center}
I valori assunti da <tipo> sono i seguenti e corrispondo ai campi definiti all'interno dell'enumerativo \textit{RequestType}.
Durante l'attuale iterazione siamo andati a gestire le richieste relative alle sole due seguenti tipologie di richieste:
\begin{itemize}
	\item 0 -> BOOK\_REGISTRATION\_MANUAL
	\item 8 -> BOOK\_SEARCH
\end{itemize}

Nella seconda iterazione, e nelle successive, è stata pianificata l'implementazione della parte di gestione delle restanti richieste:

\begin{itemize}
	\item 1 -> BOOK\_RESERVATION	
	\item 2 -> LOGIN
	\item 3 -> SIGN\_IN
	\item 4 -> BOOK\_REGISTRATION\_AUTOMATIC
	\item 5 -> PROFILE\_INFO
	\item 6 -> TAKEN\_BOOKS
	\item 7 -> PICK\_UP	
\end{itemize}



TODO: inserire esempio di una stringa realemente generata da android.


Analogamente possiamo identificare i messaggi inviati dal server come:
\begin{center}
	\begin{lstlisting}
	^[a-zA-Z0-9]+:[0-9]+;([a-zA-Z0-9]+:[a-zA-Z0-9]+;)+$
	\end{lstlisting}
\end{center}
la cui semantica può essere così rappresentata:
\begin{center}
	"requestType:<tipo>;<risultato>"
\end{center}

TODO: inserire esempio di una stringa realemente generata dal server.


Qual'ora il server riceva una richiesta con un tipo errato la risposta del server è la seguente:
\begin{center}
	\begin{lstlisting}
	"requestType:10000;result:KO_RequestType"
	\end{lstlisting}
\end{center}




