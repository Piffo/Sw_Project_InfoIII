I messaggi inviati dai client sono stringhe che possono essere identificate dalla seguente \textit{RegEx} (\textit{Regular Expression}): 
\begin{center}
	\begin{lstlisting}
	^[a-zA-Z0-9]+;[a-zA-Z0-9]+:[0-9]+;([a-zA-Z0-9]+:[a-zA-Z0-9]+;)+$
	\end{lstlisting}
\end{center}
la cui semantica può essere così rappresentata:
\begin{center}
"<username>;requestType:<tipo>;<richiesta>"
\end{center}
i valori assunti da <tipo> sono i seguenti:
\begin{itemize}
	\item 0 -> Registrazione manuale del libro
	\item 1 -> Prenotazione di un libro	
	\item 2 -> Login
	\item 3 -> Registrazione nuovo utente
	\item 4 -> Registrazione automatica del libro
	\item 5 -> Informazioni del profilo
	\item 6 -> Libri presi ???????????????????????????
	\item 7 -> Libri Raccolti ????????????????????????
	\item 8 -> Ricerca del libro
\end{itemize}
tale enumerazione è implementata da \textit{RequestType}.
TODO: inserire esempio di una stringa realemente generata da android.


Analogamente possiamo identificare i messaggi inviati dal server come:
\begin{center}
	\begin{lstlisting}
	^[a-zA-Z0-9]+:[0-9]+;([a-zA-Z0-9]+:[a-zA-Z0-9]+;)+$
	\end{lstlisting}
\end{center}
la cui semantica può essere così rappresentata:
\begin{center}
	"requestType:<tipo>;<risultato>"
\end{center}
TODO: inserire esempio di una stringa realemente generata dal server.


Qual'ora il server riceva una richiesta con un tipo errato la risposta del server è la seguente:
\begin{center}
	\begin{lstlisting}
	"requestType:10000;result:KO_RequestType"
	\end{lstlisting}
\end{center}




