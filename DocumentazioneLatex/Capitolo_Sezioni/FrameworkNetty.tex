
Netty è un NIO (Non-blocking Input Output) client server framework (i.e. permette una comunicazione asincrona tra client e server di tipo event-driven) che facilita lo sviluppo di applicazioni di rete che utilizzano socket TCP e UDP.

Lato Server creiamo un oggetto singleton \textit{Communication} che inizializza un oggetto di tipo \textit{ServerBootstrap} (io.netty.bootstrap.ServerBootstrap) il quale effettua il bootstrap del \textit{ServerChannel} ovvero del link "primario" (sulla porta 5000) che accetta ogni connessione di un client e crea un canale "figlo" sul quale avverà la comunicazione:
\begin{lstlisting}
private Communication() throws Exception {
	~
	ServerBootstrap b = new ServerBootstrap();

	b.group(bossGroup, workerGroup).channel(NioServerSocketChannel.class)
	.handler(new LoggingHandler(LogLevel.INFO)).childHandler(
		 new ServerInitializer(sslCtx));

	b.bind(PORT).sync().channel().closeFuture().sync();
	~
}
\end{lstlisting}
al server aggiungiamo un handler \textit{ServerInitializer} che aggiunge alla pipeline, oltre agli encoder e decoder di default, il \textit{ServerHandler}
\begin{lstlisting}
class ServerInitializer extends ChannelInitializer<SocketChannel> {

	private static final StringDecoder DECODER = new StringDecoder();
	private static final StringEncoder ENCODER = new StringEncoder();
	private static final ServerHandler SERVER_HANDLER = new ServerHandler();
	~
	@Override
	public void initChannel(SocketChannel ch) throws Exception {
		ChannelPipeline pipeline = ch.pipeline();
		~
		pipeline.addLast(DECODER);
		pipeline.addLast(ENCODER);
		// and then business logic.
		pipeline.addLast(SERVER_HANDLER);
	}
}
\end{lstlisting}
la classe \textit{ServerHandler} è un sottotipo di \textit{SimpleChannelInboundHandler} \\(io.netty.channel.SimpleChannelInboundHandler) il quale fornisce la stringa inviata dal client tramite il metodo:
\newpage
\begin{lstlisting}
class ServerHandler extends SimpleChannelInboundHandler<String> {
~
	@Override
	public void channelRead0(ChannelHandlerContext ctx, String request)
				throws Exception {
		int j = request.indexOf(";");
		String username = request.substring(0, j);
		Communication.chcMap.put(username, ctx);
		
		System.out.println("Process request: " + request);
		computeRequest.process(request.substring(j + 1), username);
	}
~
}
\end{lstlisting}
l'attento lettore avrà notato che il metodo precedente memorizza la coppia (username, channelHandlerContext) all'interno dell'oggetto chcMap di \textit{Communication}, il quale è di tipo HashMap<String, ChannelHandlerContext>, tale informazione è fondamentale al fine di rispondere al clinet tramite canale "figlio" ad esso dedicato e chiuderlo al termine della comunicazione; il metodo invoca l'elaborazione del messaggio ricevuto tramite l'interfaccia \textit{ProcessRequest} (la semantica della stringa request verrà discussa nel paragrafo successivo).


Infine, ma non per importanza, discutiamo l'implementazione dell'interfaccia \textit{SendAnswer} da parte della classe \textit{Communication}:
\begin{lstlisting}
public final class Communication implements SendAnswer{
	~
	public void send(final String username, final String msg) {
		~
		ChannelFuture oo = chcMap.get(username).writeAndFlush(msg + "\r\n");
		
		oo.addListener(new ChannelFutureListener() {
		
			public void operationComplete(ChannelFuture future)	
							throws Exception {
				int count = 0;
				boolean isError = false;
				
				while(!future.isSuccess()) {
					chcMap.get(username).writeAndFlush(msg + "\r\n");
					System.out.println("Retry send response!");
					if(count == 3) {
						isError = true;
						break;
					} else 
						count ++;
				}
				
				~
				
				future.addListener(ChannelFutureListener.CLOSE);
				chcMap.remove(username);
			}
		});
	}
	~
}
\end{lstlisting}
il metodo send si occupa di inviare la risposta relativa alla richiesta ricevuta e nel caso di fallimento ripeterà l'invio fino a un limite di 3 tentativi.
\newpage


