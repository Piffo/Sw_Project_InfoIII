Le seguenti sono le query ddl per la creazione del database che abbiamo utilizzato:
\begin{lstlisting}
-- tables
-- Table: Libro
CREATE TABLE Libro (
BCID varchar2(15)  NOT NULL,
Titolo varchar2(256)  NOT NULL,
Autore varchar2(256)  NOT NULL,
DataPubblicazione date  NULL,
ISBN varchar2(13)  NOT NULL,
Propietario varchar2(20)  NOT NULL,
CONSTRAINT Libro_pk PRIMARY KEY (BCID)
) ;

-- Table: Possesso
CREATE TABLE Possesso (
Utente varchar2(20)  NOT NULL,
Libro varchar2(15)  NOT NULL,
DataInizio date  NOT NULL,
DataFine date  NOT NULL,
LuogoRilascio varchar2(512)  NOT NULL,
CONSTRAINT Possesso_pk PRIMARY KEY (Utente,Libro,DataInizio)
) ;

-- Table: Prenotazione
CREATE TABLE Prenotazione (
Utente varchar2(20)  NOT NULL,
Libro varchar2(15)  NOT NULL,
CONSTRAINT Prenotazione_pk PRIMARY KEY (Utente,Libro)
) ;

-- Table: Utente
CREATE TABLE Utente (
Username varchar2(20)  NOT NULL,
Nome varchar2(50)  NOT NULL,
Cognome varchar2(50)  NOT NULL,
DataNas date  NOT NULL,
Contatto varchar2(256)  NULL,
Password char(256)  NOT NULL,
ResidenzaLat integer  NOT NULL,
ResidenzaLong integer  NOT NULL,
RaggioAzione integer  NOT NULL,
CONSTRAINT Utente_pk PRIMARY KEY (Username)
) ;

-- foreign keys
-- Reference: Libro_Utente (table: Libro)
ALTER TABLE Libro ADD CONSTRAINT Libro_Utente
FOREIGN KEY (Propietario)
REFERENCES Utente (Username);

-- Reference: Possiede_Libro (table: Possesso)
ALTER TABLE Possesso ADD CONSTRAINT Possiede_Libro
FOREIGN KEY (Libro)
REFERENCES Libro (BCID);

-- Reference: Possiede_Profilo (table: Possesso)
ALTER TABLE Possesso ADD CONSTRAINT Possiede_Profilo
FOREIGN KEY (Utente)
REFERENCES Utente (Username);

-- Reference: Prenotazione_Libro (table: Prenotazione)
ALTER TABLE Prenotazione ADD CONSTRAINT Prenotazione_Libro
FOREIGN KEY (Libro)
REFERENCES Libro (BCID);

-- Reference: Prenotazione_Utente (table: Prenotazione)
ALTER TABLE Prenotazione ADD CONSTRAINT Prenotazione_Utente
FOREIGN KEY (Utente)
REFERENCES Utente (Username);
\end{lstlisting}

TODO: aggiungere creazione tabella PASSAGGIO
