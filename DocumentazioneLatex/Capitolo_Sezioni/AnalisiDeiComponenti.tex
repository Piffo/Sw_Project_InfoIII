Come già presentato in precedenza, per la definizione dei componenti si è deciso di seguire il pattern architetturale MVC. 
Le componenti che si è deciso di sviluppare durante la prima iterazione sono: 
\begin{itemize}
	\item \textbf{Componente \textit{Manual Book registration:}} La componente \textit{Manual Book registration} fa riferimento all’ UC9 (~\ref{itemize:UC9} ) (figlio del caso d'uso più generico UC5 (~\ref{itemize:UC5} ), ovvero alla funzione di aggiunta di un libro alla rete di Book Crossing per via manuale. La componente si presenta nel seguente modo:
	\begin{itemize}
		\item \textit{GUI:} Interfaccia grafica utilizzata per registrare un libro alla rete di Book Crossing. Verranno quindi messe a disposizione una serie di interfaccie grafiche, composte sostanzialmente da campi da compilare, per aggiungere le informazioni relative al proprio libro, ottenendo poi, successivamente alla registrazione, il relativo BCID;
		\item \textit{Model:} Si fa carico di ricevere le informazioni relative al libro e, sfruttando la parte Data, restituisce alla parte GUI il BCID con il quale siglare il libro;
		\item \textit{Data:} Le informazioni relative al libro che si vuole aggiungere sono memorizzate nel Database RDS, associandolo all'utente che attualmente lo possiede. 
	\end{itemize}
	\item \textbf{Componente \textit{Book reservation:}}  La componente \textit{Book reservation} fa riferimento all’ U13 (~\ref{itemize:UC13} ), ovvero alla funzione di prenotazione di un libro, la quale richiede prima un'operazione di ricerca dello stesso all'interno della community, e poi, se possibile, permette di effettuare la prenotazione  effettiva. La componente si presenta nel seguente modo:
	\begin{itemize}
		\item \textit{GUI:} Interfaccia grafica utilizzata per poter procedere con la prenotazione di un libro della rete di Book Crossing. In questo caso sarà possibile compiere tale azione attraverso una procedura di ricerca del libro, oppure navigando nella propria sezione personale del profilo;
		\item \textit{Model:} Si fa carico di gestire la coda relativa alla prenotazione di un determinato libro, in modo da poterle soddisfare. La gestione e la computazione della prenotazione è affidata ad un algoritmo il quale va ad appoggiarsi alla parte Data per poter risalire ai dati dei richiedenti;
		\item \textit{Data:} Le informazioni relative al libro che si vuole prenotare e agli utenti richiedenti le quali sono memorizzate nel Database.
	\end{itemize}
\end{itemize}