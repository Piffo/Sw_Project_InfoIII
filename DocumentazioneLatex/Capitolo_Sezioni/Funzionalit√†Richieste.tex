\begin{table}
\caption{Panoramica requisiti funzionali progettuali}
\label{tab:Req_utente}
\begin{tabular}{|c|c|c|c|c|c|c}
	\begin{tabular}{p{3cm}|c|c|c|c|c}
		\hline\hline
		\textbf{Nome requisito} & \textbf{ID requisito} & \textbf{Tipologia} & \textbf{Priorità} & \textbf{Requisiti padre} & \textbf{Requisiti figli} \\
		\hline\hline
		Raccolta libro & UR1 & funzionale & alta & & UR2\\
		\hline
		Login utente & UR2 & funzionale & alta & UR1 & UR3, UR4\\
		\hline
		Registrazione utente & UR3 & funzionale & alta & UR2 & \\
		\hline
		Aggiunta libro & UR4 & funzionale & alta & UR2 & UR7, UR9 \\
		\hline
		Ricerca libro & UR5 & funzionale & media & UR2 & UR10 \\
		\hline
		Prenotazione libro & UR6 & funzionale & bassa & UR2 &\\
		\hline
		Visualizzazione info libri chased & UR7 & funzionale & bassa & UR2 &\\
		\hline
		Visualizzazione info libri released & UR8 & funzionale & bassa & UR2, UR4 &\\
		\hline
		Rilascio libro & UR9 & funzionale & alta & UR2, UR4 &\\
		\hline
		Visualizzazione contatti utenti & UR10 & funzionale & bassa & UR2, UR5 &\\
		\hline
		Visualizzazione profilo personale  & UR11 & funzionale & media & UR2 &\\
		\hline
	\end{tabular}
\end{tabular}
\end{table}

\begin{table}

	\caption{Descrizione requisiti funzionali progettuali}
	\label{tab:Req_utente_descrizione}
	\centering
	\begin{tabular}{|p{5cm}|c|p{5cm}|}
		\hline\hline
		\textbf{Nome requisito} & \textbf{ID requisito} & \textbf{Descrizione} \\
		\hline\hline
		Controllo geolocalizzazione & UR12 & Requisito non funzionalità che permette di verificare che la posizione GPS
		salvata del libro corrisponda, con margine d'accettazione, alla posizione 
		in cui si trova l'utente nel momento in cui vuole raccogliere
		un libro trovato "on the go".\\
		\hline
	\end{tabular}
\end{table}