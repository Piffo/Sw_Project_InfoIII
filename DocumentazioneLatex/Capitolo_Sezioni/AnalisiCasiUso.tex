

\begin{itemize}
	\item \textbf{\textit{UC1: Registration}}
	\begin{itemize}
		\item \textbf{Descrizione:} registrazione alla rete di Book Crossing.
		\item \textbf{Attori coinvolti:} utente.
		\item \textbf{Preconditions:} 
		\begin{itemize}
			\item smartphone dotato di connessione dati;
			\item l’utente non è ancora registrato al programma di Book Crossing.
		\end{itemize}
		\item \textbf{Postconditions:} l’utente è registrato al programma di Book Crossing.
		\item \textbf{Processo:}
		\begin{enumerate}
			\item l’utente seleziona “Registrati” nella schermata inziale dell’applicazione;
			\item l’applicazione mostra un form in cui l'utente può inserire:
			\begin{enumerate}
				\item [a.] Nome
				\item [b.] Cognome
				\item [c.] Data di nascita
				\item [d.] Username
				\item [e.] Password
				\item [f.] Contatto di riferimento
				\item [g.] Categorie di libri preferite
				\item [h.] Zona di residenza
				\item [i.] Raggio d'azione rispetto alla zona di residenza
			\end{enumerate}
			\item l’utente inserisce i dati richiesti nel form presentato;
			\item l'utente attende la visualizzazione della conferma di avvenuta registrazione;
			\item l’utente viene reindirizzato alla pagina principale dell’applicazione.
		\end{enumerate}
		\item \textbf{Alternative}
		\begin{itemize}
			\item \textbf{Dati non validi:} se l'utente inserisce dei dati non validi e/o mancanti, l'applicazione mostra un messaggio d'errore permettendo all'utente di modificare i dati non validi.
		\end{itemize}
		\item \textbf{Estensioni}
	\end{itemize}
	\item \textbf{\textit{UC2: Login}}
	\begin{itemize}
		\item \textbf{Descrizione:} accesso alla rete di Book Crossing.
		\item \textbf{Attori coinvolti:} utente. 
		\item \textbf{Preconditions:} 
		\begin{itemize}
			\item smartphone dotato di connessione dati;
			\item l’utente è già registrato al programma di Book Crossing.
		\end{itemize}
		\item \textbf{Postconditions:} l’utente è loggato nella rete di Book Crossing.
		\item \textbf{Processo:}
		\begin{enumerate}
			\item l’utente seleziona “Login” nella schermata iniziale dell’applicazione;
			\item l’applicazione mostra una schermata in cui l'utente inserisce username e password;
			\item l’utente inserisce i dati richiesti nella view presentata;
			\item l'utente attende la verifica della correttezza dei dati inseriti;
			\item l’utente viene reindirizzato alla pagina principale dell’applicazione.
		\end{enumerate}
		\item \textbf{Alternative}
		\begin{itemize}
			\item \textbf{Username e/o password non corretti:}  se l'utente inserisce username e/o password non validi, l'applicazione mostra un messaggio d'errore permettendo all'utente di modificare i dati.
		\end{itemize}
		\item \textbf{Estensioni}
	\end{itemize}
		\item \textbf{\textit{UC3: Logout}}
	\begin{itemize}
		\item \textbf{Descrizione:} disconnessione profilo personale.
		\item \textbf{Attori coinvolti:} utente. 
		\item \textbf{Preconditions:} 
		\begin{itemize}
			\item smartphone dotato di connessione dati;
			\item l’utente è già registrato al programma di Book Crossing;
			\item l'utente è loggato.
		\end{itemize}
		\item \textbf{Postconditions:} l’utente non è più loggato nella rete di Book Crossing.
		\item \textbf{Processo:}
		\begin{enumerate}
			\item l’utente seleziona “Profilo personale” nella schermata principale dell’applicazione;
			\item l’applicazione mostra una schermata in cui l'utente può visualizzare tutte le proprie informazioni;
			\item l’utente preme il bottone "Logout";
			\item l’utente viene reindirizzato alla pagina di login.
		\end{enumerate}
	\end{itemize}

	\item \textit{\textbf{UC4: Book pick-up}}
	\begin{itemize}
		\item \textbf{Descrizione:} raccolta di un libro “on the go”\footnote{In questo caso il libro condiviso dalla community viene raccolto dall'utente in una qualsiasi zona (come per esempio la stazione, il parco, sale di attesa etc...).} o in una OCZ.\footnote{\textit{Official Crossing Zone}: zone riconosciute e fisse in cui la community può liberamente scambiarsi i libri.}
		\item \textbf{Attori coinvolti:} utente.
		\item \textbf{Preconditions:}
		\begin{itemize}
			\item smartphone dotato di connessione dati;
			\item l’utente ha effettuato l’accesso alla rete di Book Crossing;
			\item il libro è stato siglato con il codice BCID.
		\end{itemize}
		\item \textbf{Postconditions:} Il libro viene associato all’utente.
		\item \textbf{Processo:}
		\begin{enumerate}
			\item l’utente seleziona “Raccogli libro” nel menu principale dell’applicazione;
			\item l’applicazione mostra un form in cui l’utente può inserire il BCID del libro;
			\item l’utente inserisce il codice BCID riportato nel libro;
			\item l'utente attende la visualizzazione di una scheda riepilogativa relativa al libro appena aggiunto;
			\item l’utente verifica la corrispondenza delle informazioni mostrate;
			\item l'utente conferma la raccolta.
		\end{enumerate}
		\item \textbf{Alternative}
		\begin{itemize}
			\item \textbf{BCID inesistente:} se l'utente inserisce un BCID non esistente, l'applicazione mostra un messaggio d'errore permettendo all'utente di modificare il BCID.
			\item \textbf{BCID associato ad un altro utente:} se l'utente inserisce un BCID già in possesso di un'altro utente, l'applicazione mostra un messaggio d'errore permettendo all'utente di modificare il BCID.
			\item \textbf{BCID non corrispondente:} se l'utente, al punto (4), verifica che il libro reale non corrisponde alle informazioni mostrate dall'applicazione, può annullare l'operazione di raccolta.
		\end{itemize}
		\item \textbf{Estensioni}
	\end{itemize}
	\item \textbf{\textit{UC5: Book registration}}
	\begin{itemize}
		\item \textbf{Descrizione:} registrazione di un libro alla rete di Book Crossing (\textit{journal entry}).
		\item \textbf{Generalizzazione di:} 
		\begin{itemize}
			\item aggiunta manuale dei dati del libro (UC8);
			\item scansione ISBN (UC9).
		\end{itemize}
		\item \textbf{Include:} scrittura BCID (UC10).
		\item \textbf{Attori coinvolti:} utente.
		\item \textbf{Preconditions:}
		\begin{itemize}
			\item smartphone dotato di connessione dati;
			\item l’utente ha effettuato l’accesso alla rete di Book Crossing;
			\item il libro non è ancora stato siglato con il codice BCID.
		\end{itemize}
		\item \textbf{Postconditions:} Il libro viene registrato alla rete di Book Crossing.		
		\item \textbf{Processo:} 
		\begin{enumerate}
			\item l’utente seleziona “Registra un nuovo libro” nel menu principale dell’applicazione;
			\item l’applicazione tenta di iniziare la scansione ISBN mostrando all'utente la fotocamera;
			\item l’utente inquadra il codice ISBN del libro per il tempo sufficiente al riconoscimento del codice stesso;
			\item l’applicazione mostra all'utente la schermata contente tutti le informazioni del libro; 
			\item l'utente preme il pulsante "Conferma registrazione", dopo aver verficato rapidamente la coerenza dei dati.
		\end{enumerate}
		\item \textbf{Alternative}
		\begin{itemize}
			\item \textbf{Aggiunta manuale dei dati:} se, dalla schermata di scansione, l'utente decide di inserire manualmente i dati del libro da registrate, l'applicazione mostra una schermata dove aggiungere manualmente i dati del libro.
			\item \textbf{Scansione fallita:} se la scansione fallisce, l'applicazione riapre la fotocamenra permettendo all'utente di ripetere l'operazione.
		\end{itemize}
		\item \textbf{Estensioni}
	\end{itemize}
	\item \textbf{\textit{UC6: Book research}}
	\begin{itemize}
		\item \textbf{Descrizione:} ricerca di un libro all’interno della rete di Book Crossing.
		\item \textbf{Attori coinvolti:} utente.
		\item \textbf{Preconditions:}
		\begin{itemize}
			\item smartphone dotato di connessione dati;
			\item l’utente ha effettuato l’accesso alla rete di Book Crossing;
			\item il libro è presente nella rete di Book Crossing.
		\end{itemize}
		\item \textbf{Postconditions:} il libro ricercato viene mostrato.
		\item \textbf{Processo:}
		\begin{enumerate}
			\item l’utente seleziona “Ricerca libro” nel menu principale dell’applicazione;
			\item l’applicazione mostra all'utente un form da completare in cui inserire i parametri della ricerca;
			\item l’utente inserisce i parametri a cui è interessato;
			\item l'utente preme il pulsante "Cerca" ed attende il completamento;
			\item l'utente visualizza la lista di tutti i libri nella rete, che soddisfano la ricerca.
			\item il sistema verifica la presenza del libro cercato;
			\item in caso di esito positivo, l’applicazione mostra una scheda riassuntiva del libro;
			\item in caso di esito negativo, l’applicazione mostrerà un messaggio di errore.
		\end{enumerate}
		\item \textbf{Alternative}
		\begin{itemize}
			\item \textbf{Parametri non validi:} se l'utente inserisce dei parametri non validi, l'applicazione mostra un messaggio d'errore permettendo all'utente di modificarli.
			\item \textbf{Ricerca senza risultati:} se la ricerca non va a buon fine, l'applicazione mostra un messaggio all'utente, comunicando che nessun libro presente nella rete soddisfa i parametri di ricerca inseriti.
		\end{itemize}
		\item \textbf{Estensioni}
		\begin{itemize}
			\item L'utente può selezionare uno dei libri mostrati dall'applicazione e visualizzare le sue informazioni.
		\end{itemize}
	\end{itemize}
	\item \textbf{\textit{UC7: Info visualization}}
	\begin{itemize}
		\item \textbf{Descrizione: } Visualizzazione informazioni
		\item \textbf{Generalizzazione:} visualizzazione informazioni libri chased (UC13), visualizzazione informazioni libri relase d(UC14) e visualizzazione informazioni libri in possesso (UC15).
		\item \textbf{Attori coinvolti:} utente.
		\item \textbf{Preconditions:}
		\begin{itemize}
			\item smartphone dotato di connessione dati;
			\item l’utente ha effettuato l’accesso alla rete di Book Crossing.
		\end{itemize}
		\item \textbf{Postconditions:} L’applicazione mostra le informazioni desiderate
		\item \textbf{Processo:}
		\begin{enumerate}
			\item l’utente seleziona la voce “I miei libri” nel menu principale dell’applicazione;
			\item l’applicazione mostra categorie di informazioni visualizzabili;
			\item l’utente seleziona la categoria che vuole visualizzare;
			\item l'applicazione mostra l'elenco dei libri della categoria selezionata.
		\end{enumerate}
		\item \textbf{Alternative}
		\begin{itemize}
			\item \textbf{Nessun libro in elenco:} se nessun libro è presente nello storico, l'applicazione mostra un messaggio all'utente, comunicando che non è stata ancora effettuata nessuna operazione nella comunità.
		\end{itemize}
		\item \textbf{Estensioni}
	\end{itemize}
	\item \textbf{\textit{UC8: Personal area visualization}}
	\begin{itemize}
		\item \textbf{Descrizione: } Visualizzazione profilo personale utente
		\item \textbf{Attori coinvolti:} utente.
		\item \textbf{Preconditions:}
		\begin{itemize}
			\item smartphone dotato di connessione dati;
			\item l’utente ha effettuato l’accesso alla rete di Book Crossing;
		\end{itemize}
		\item \textbf{Postconditions: }l’applicazione mostra il profilo dell’utente.
		\item \textbf{Processo: }
		\begin{enumerate}
			\item l’utente seleziona la voce “Il mio profilo” nel menu principale dell’applicazione;
			\item l’applicazione mostra l’anagrafica, i contatti e le attività svolte dall’utente;
		\end{enumerate}
		\item \textbf{Estensioni:}
	\end{itemize}
	\item \textbf{\textit{UC9: Manual addition of book's data}}
	\begin{itemize}
		\item \textbf{Descrizione:} Inserimento manuale di un libro nella rete di Book Crossing
		\item \textbf{Attori coinvolti:} utente.
		\item \textbf{Preconditions:}
		\begin{itemize}
			\item smartphone dotato di connessione dati;
			\item l’utente ha effettuato l’accesso alla rete di Book Crossing;
			\item il libro non è stato ancora siglato con il codice BCID.
		\end{itemize}
		\item \textbf{Postcondition:} Viene generato il codice BCID e il libro viene aggiunto alla rete di Book Crossing
		\item \textbf{Processo: }
		\begin{enumerate}
			\item facendo riferimento al passo 1 e 2 del UC4, l'utente preme il pulsante “Aggiunta manuale”;
			\item l’applicazione mostra un form da compilare con i dati del libro;
			\item l’utente inserisce i dati del libro richiesti e conferma l’operazione;
			\item l’applicazione mostra il codice BCID da trascrivere sul libro;
			\item il sistema aggiunge il libro alla rete di Book Crossing.
		\end{enumerate}
		\item \textbf{Estensioni}
	\end{itemize}
	\item \textbf{\textit{UC10: ISBN scan}}
	\begin{itemize}
		\item \textbf{Descrizione:} scansione del codice ISBN tramite fotocamera per ottenere le informazioni in merito al libro da registrare.
		\item \textbf{Attori coinvolti:} utente.
		\item {Preconditions:} 
		\begin{itemize}
			\item smartphone dotato di connessione dati;
			\item l’utente ha effettuato l’accesso alla rete di Book Crossing;
			\item il libro possiede il codice ISBN.
		\end{itemize}
		\item \textbf{Postconditions:} il libro è in possesso dell'utente e non più condiviso con la community.
		\item \textbf{Processo:}
		\begin{enumerate}
			\item l’utente seleziona “Registra un nuovo libro” nel menu principale dell’applicazione;;
			\item Viene aperta la fotocamera all'interno dell'applicazione;
			\item l'utente inquadra il codice ISBN finchè il sistema non rileva il barcode.
		\end{enumerate}
		\item \textbf{Alternative}
		\begin{itemize}
			\item \textbf{ISBN non riconsciuto:} il sistema non è in grado di riconoscere l'ISBN inquadrato. Si chiuderà la fotocamera e l'utente verrà reindirizzato alla pagina di inserimento manuale del libro  (UC9).
		\end{itemize}
		\item \textbf{Estensioni}
	\end{itemize}
	\item \textbf{\textit{UC11: Instruction to write BCID code\footnote{\textit{Book Crossing IDentifier}}}}
	\begin{itemize}
		\item \textbf{Descrizione:} scrittura del codice identificativo sul libro condiviso.
		\item \textbf{Attori coinvolti:} utente
		\item \textbf{Preconditions:}
		\begin{itemize}
			\item smartphone dotato di connessione dati;
			\item l’utente ha effettuato l’accesso alla rete di Book Crossing;
		\end{itemize}
		\item \textbf{Postconditions:} il libro è univocamente riconosciuto del sistema tramite il BCID.
		\item \textbf{Processo:} l'utente copia il codice BCID sul libro, seguendo le istruzioni mostrate dall'applicazione.
		\item \textbf{Estensioni}
	\end{itemize}
	\item \textbf{\textit{UC12: View of users contacts}}
	\begin{itemize}
		\item \textbf{Descrizione:} l'utente ottiene i contatti che un altro utilizzatore ha deciso di condividere con la community.
		\item \textbf{Attori coinvolti:} utente.
		\item \textbf{Preconditions:}
		\begin{itemize}
			\item smartphone dotato di connessione dati;
			\item l’utente ha effettuato l’accesso alla rete di Book Crossing.
		\end{itemize}
		\item \textbf{Postconditions:} l'applicazione visualizza i contatti a cui l'utente può e/o vuole essere contattato.
		\item \textbf{Processo:}
		\begin{enumerate}
			\item l'utente selezione "Ricerca" dal menu principale dell'applicazione;
			\item l'utente seleziona il libro a cui è interessato;
			\item il sistema mostra tutte le informazioni relative al libro (tra cui anche la lista di tutti i lettori che sono stati in possesso del libro in questione);
			\item l'utlizzatore seleziona l'utente che desidera contattare;
			\item il sistema mostra tutte le informazioni di contatto che il cliente terzo ha deciso di condividere con la community.
		\end{enumerate}
		\item \textbf{Alternative:}
		\begin{itemize}
			\item \textbf{Nessun libro trovato:} l'applicazione notifica l'utilizzatore del fatto che la ricerca non sia andata a buon fine.
			\item \textbf{Utente senza alcun contatto condiviso:} il sistema filtra la visualizzazione della lista degli utenti, visualizzando solo coloro che hanno inserito, durante la fase di registrazione, almeno un contatto o che desiderano essere contattati.
		\end{itemize}
		\item \textbf{Estensioni}
	\end{itemize}
	\item \textbf{\textit{UC13: Book reservation}}
	\begin{itemize}
		\item \textbf{Descrizione:} l'utilizzatore prenota un determinato libro in possesso di un altro lettore.
		\item \textbf{Attori coinvolti:}
		\begin{itemize}
			\item utente richiedente (\textit{claimant user});
			\item utente attualmente in possesso del libro richiesto (\textit{owner user}).
		\end{itemize}
		\item \textbf{Preconditions:}
		\begin{itemize}
			\item smartphone dotato di connessione dati;
			\item l’utente ha effettuato l’accesso alla rete di Book Crossing;
			\item il libro che si vuole prenotare deve essere registrato alla rete;
			\item il libro richiesto deve essere già in possesso di un altro utente.
		\end{itemize}
		\item \textbf{Postconditions:}
		\begin{itemize}
			\item Se i due utenti hanno un punto di incontro in comune, si accordano sul luogo di scambio. 
			\item Se invece non hanno un punto di incontro comune, il libro passerà tra gli utenti che si trovano tra \textit{claimant user} e \textit{owner user}.
		\end{itemize}
		\item \textbf{Processo:}
		\begin{enumerate}
			\item l’utente seleziona “Ricerca libro” nel menu principale dell’applicazione;
			\item l’applicazione mostra all'utente un form da completare in cui inserire i parametri della ricerca;
			\item l’utente inserisce i parametri a cui è interessato;
			\item l'utente preme il pulsante "Cerca" ed attende il completamento;
			\item l'utente visualizza la lista di tutti i libri nella rete, che soddisfano la ricerca.
			\item il sistema verifica la presenza del libro cercato;
			\item l'utente va a selezionare il libro all'interno della lista proposta dal sistema;
			\item l'applicazione mostra un riepilogo sulle informazioni del libro, unitamente alla possibilità di prenotare;
			\item l'utente preme il pulsante "Prenota";
			\item l'applicazione mostra una pagina di conferma dell'avvenuta prenotazione.
		\end{enumerate}
		\item \textbf{Alternative}
		\begin{itemize}
			\item \textbf{Libro non prenotabile:} il libro selezionato non è prenotabile poichè non in possesso di un altro utente.
		\end{itemize}
		\item \textbf{Estensioni}
	\end{itemize}
	\item \textbf{\textit{UC14: Chased books informations}}
	\begin{itemize}
		\item \textbf{Descrizione:} visualizzazione storico dei libri "raccolti" dall'utente.
		\item \textbf{Attori coinvolti:} utente.
		\item \textbf{Preconditions:}
		\begin{itemize}
			\item smartphone dotato di connessione dati;
			\item l’utente ha effettuato l’accesso alla rete di Book Crossing.
		\end{itemize}
		\item \textbf{Postconditions:} mostrata sulla grafica la lista dei libri raccolti, con relativa data e luogo di "chasing".
		\item \textbf{Processo:}
		\begin{enumerate}
			\item l'utente seleziona "Il mio profilo" dal menù principale dell'applicazione;
			\item l'applicazione mostra le informazioni dell'utente e i diversi stati in cui si possono trovare i suoi libri;
			\item l'utente sceglie lo stato "Libri chased";
			\item l'applicazione mostra un riepilogo di tutti i testi ottenuti dalla community di sharing.
		\end{enumerate}
		\item \textbf{Alternative:}
		\begin{itemize}
			\item \textbf{Nessun libro chased:} l'applicazione mostra un messaggio all'utente, comunicando che nessun libro è stato ancora raccolto da lui dalla community.
		\end{itemize}
		\item \textbf{Estensioni}
	\end{itemize}
	\item \textbf{\textit{UC15: Released books informations}}
	\begin{itemize}
		\item \textbf{Descrizione:} visualizzazione storico dei propri libri condivisi con la rete.
		\item \textbf{Attori coinvolti:}  utente.
		\item \textbf{Preconditions:}
		\begin{itemize}
			\item smartphone dotato di connessione dati;
			\item l’utente ha effettuato l’accesso alla rete di Book Crossing.
		\end{itemize}
		\item \textbf{Postconditions:} mostrata sulla grafica la lista dei libri inseriti nel programma di sharing, con relativa data e luogo di "relase".
		\item \textbf{Processo:}
		\begin{enumerate}
			\item l'utente seleziona "Il mio profilo" dal menù principale dell'applicazione;
			\item l'applicazione mostra le informazioni dell'utente e i diversi stati in cui si possono trovare i suoi libri;
			\item l'utente sceglie lo stato "Libri relased";
			\item l'applicazione mostra un riepilogo di tutti i libri rilasciati alla community di sharing.
		\end{enumerate}
		\item \textbf{Alternative:}
		\begin{itemize}
			\item \textbf{Nessun libro released:} l'applicazione mostra un messaggio all'utente, comunicando che nessun libro è stato ancora rilasciato da lui nella rete di Book Crossing.
		\end{itemize}
		\item \textbf{Estensioni}
	\end{itemize}
	\item \textbf{\textit{UC16: "Under reading" books informations}}
	\begin{itemize}
		\item \textbf{Descrizione:} visualizzazione dei propri libri attualmente "under reading".
		\item \textbf{Attori coinvolti:} utente.
		\item \textbf{Preconditions:}
		\begin{itemize}
			\item smartphone dotato di connessione dati;
			\item l’utente ha effettuato l’accesso alla rete di Book Crossing.
		\end{itemize}
		\item \textbf{Postconditions:} l'applicazione mostra la lista dei libri attualmente in possesso dell'utente 
		\item \textbf{Processo:}
		\begin{enumerate}
			\item l'utente seleziona "Il mio profilo" dal menù principale dell'applicazione;
			\item l'applicazione mostra le informazioni dell'utente e i diversi stati in cui si possono trovare i suoi libri;
			\item l'utente sceglie lo stato "Libri in possesso";
			\item l'applicazione mostra un riepilogo di tutti i libri attualmente in possesso.
		\end{enumerate}
		\item \textbf{Alternative:}
		\begin{itemize}
			\item \textbf{Nessun libro in lettura:} l'applicazione mostra un messaggio all'utente, comunicando che nessun libro è in suo possesso al momento.
		\end{itemize}
		\item \textbf{Estensioni}
	\end{itemize}
	\item \textbf{\textit{UC17: Book relased}}
	\begin{itemize}
		\item \textbf{Descrizione:} l'utente libera un libro.
		\item \textbf{Attori coinvolti:} utente.
		\item \textbf{Preconditions:}
		\begin{itemize}
			\item smartphone dotato di connessione dati;
			\item l’utente ha effettuato l’accesso alla rete di Book Crossing;
			\item libro rilasciato già registrato al sistema di Book Crossing.
		\end{itemize}
		\item \textbf{Postconditions:} il libro passa dallo stato "under reading" a quello "Available".
		\item \textbf{Processo:}
		\begin{enumerate}
			\item l'utente seleziona "Il mio profilo" dal menù principale dell'applicazione;
			\item l'applicazione mostra le informazioni dell'utente e i diversi stati in cui si possono trovare i suoi libri;
			\item l'utente sceglie lo stato "Libri in possesso";
			\item l'applicazione mostra un riepilogo di tutti i libri attualmente in possesso;
			\item l'utente preme sul testo che intende rilasciare;
			\item l'utente conferma il rilascio.
			\item l'applicazione mostra una conferma di avvenuto rilascio del testo selezionato.
		\end{enumerate}
		\item \textbf{Alternative:}
		\begin{itemize}
			\item \textbf{Nessun libro in lettura:} l'applicazione mostra un messaggio all'utente, comunicando che nessun libro è in suo possesso al momento.
			\item \textbf{Segnale GPS non trovato:} l'applicazione avvisa l'utente di attivare il GPS del dispositivo.
		\end{itemize}
		\item \textbf{Estensioni}
	\end{itemize}
\end{itemize}